\section{Zusammenfassung}

In den Kapiteln zuvor wurden die Vier Streaming framework Apache Storm, Apache Kafka, Apache Flume und Apache S4 in der Architktur, Installation und Entwicklung vorgestellt. Die in dieser Arbeit abgehandelten Streaming frameworks sind in einer speziellen Umgebung aufgebaut und haben wenige Überschneidungen in der Verwendung gleicher Datenflussverarbeitung. Eine kurze Aufzählung der wesentlichen Kernelementen jeder Streaming frameworks soll das Wissen auffrischen und für die folgenden Kapitel vorbereiten.

Ein Apache Zookeeper Cluster wird in allen Streaming frameworks außer in Apache Flume für die Synchronisierung von verteilten Prozessen verwendet. 
Apache Storm benutzt einen \textit{Task}-\textit{Scheduler} zur Arbeitsverteilung und stellt mehrere Primitive, Operatoren und Funktionen für die direkte und mit Trident für die transaktionssichere Datenverarbeitung bereit. Bei Apache Kafka kommt das \textit{Publisher}-\textit{Subscriber} Nachrichtenmuster für die Verteilung der Nachrichten zum Einsatz. Dabei wird auf einem \textit{Topic} jeweils die Nachricht gelegt und von einem \textit{Consumer} abgeholt. Apache Flume hingegen kommuniziert über konfigurierte Direktverbindungen in einem Client-Server-Modell. Auch in Apache Flume werden, wie in Apache Kafka, Nachrichten über \textit{Channels} ausgetauscht. Eine \textit{Source} sendet in einen \textit{Channel} und eine \textit{Sink} wird über eine Nachricht informiert. Gegenüber Apache Kafka werden mehrere \textit{Sources}-, \textit{Channel}- und \textit{Sink}-Typen bereitgestellt. Apache S4 benutzt zum Austausch der Nachrichten einen \textit{Communication Layer} und kommt ohne einen Master aus. Es gibt einen Typen das Processing Element, das weiter spezifiziert werden kann. Für das Monitoring kann in allen Streaming frameworks \gls{glo:jmx} benutzt werden. Die Entwicklung von eigenen Komponenten kann in der Programmiersprache Java erfolgen. Für den Vergleich der Streaming frameworks werden die gewonnen Erkenntnisse aus den Kapiteln \ref{chapter:grundlagen}, \ref{chapter:kriterien} und \ref{chapter:vorstellung} im nächsten Kapitel zur Entwicklung eines Prototypen herangezogen. Zunächst werden die einzelnen Implementierungen der Streaming frameworks für die Performanzmessung vorgestellt und beschrieben.