\section{Installationsanleitung Apache Flume}
\label{section:flumeinstall}


In diesem Kapitel wird die Installation von Apache Flume schrittweise gezeigt. Für die Installation von Apache Flume wird die aktuelle Version 1.5.0.1 verwendet. Als Betriebssystem wird Linux mit der Distribution Debian 7 eingesetzt. Eine Kompilation der Anwendung Apache Flume wird nicht durchgeführtm, daher werden einfache Kenntnisse in der Administration und Pflege von Linux-Betriebssystemen in der Shell Bash vorausgesetzt. Nach der Installation wird eine kleine Anwendung eines Apache Flume Agenten vorgestellt. 

Für den Einsatz von Apache Flume ist eine Java-Umgebung-Laufzeitumngebung notwendig. Die Installation der Debian Java7-Pakets kann mit Werkzeug \textit{aptitude} oder \textit{apt-get} durchgeführt werden.

\begin{verbatim}
aptitude install openjdk-7-jdk	
\end{verbatim}

Anschließend wird die vorkompilierte Apache Flume Anwenundung als Archiv vom Apache Server in das Verzeichnis \textit{/opt} heruntergeladen und entpackt. Abschließend wird das entpackte Verzeichnis in den Namen \textit{flume} umbenannt:

\begin{verbatim}
cd /opt
wget http://www.apache.org/dist/flume/1.5.0.1/apache-flume-1.5.0.1-bin.tar.gz
tar xvfz apache-flume-1.5.0.1-bin.tar.gz
mv apache-flume-1.5.0.1-bin.tar.gz flume
\end{verbatim}

\subsection{Apache Flume Konfiguration}
\label{section:flumeKonfiguration}

Damit Apache Flume ordentlich ausgeführt wird, muss die Konfiguration von Apache Flume für die Java-Umgebung und die JAVA\_HOME-Variable gesetzt sein.
Im Unterverzeichnis conf befinden sich template-Konfigurationsdateien. Zuerst werden die Template-Dateien in Konfigurationsdateien kopiert und anschließend bearbeitet:

\begin{verbatim}
cp flume-env.sh.template flume-env.sh
cp flume-conf.properties.template flume-conf.properties
\end{verbatim}

Mit einem Texteditor wie zum Beispiel vim, emacs oder nano können Dateien bearbeitet werden:
\begin{verbatim}
vim flume-env.sh
\end{verbatim}

Die folgende Konfigurationsdatei \textit{flume-env.sh} stellt ein Beispiel für Apache Flume dar. Zum einen wird die JAVA\_HOME-Variable mit dem richtigen Pfad zum Java-Installationsort gesetzt und zum Anderen wird das Moninoring über JMX mit Optionen für den Java Heap bereitgestellt.

\begin{lstlisting}[language=BASH, label=lst:flumeEnv, caption=Apache Flume Konfiguration]
# Licensed to the Apache Software Foundation (ASF) under one
# or more contributor license agreements.  See the NOTICE file
# distributed with this work for additional information
# regarding copyright ownership.  The ASF licenses this file
# to you under the Apache License, Version 2.0 (the
# "License"); you may not use this file except in compliance
# with the License.  You may obtain a copy of the License at
#
#     http://www.apache.org/licenses/LICENSE-2.0
#
# Unless required by applicable law or agreed to in writing, software
# distributed under the License is distributed on an "AS IS" BASIS,
# WITHOUT WARRANTIES OR CONDITIONS OF ANY KIND, either express or implied.
# See the License for the specific language governing permissions and
# limitations under the License.

# If this file is placed at FLUME_CONF_DIR/flume-env.sh, it will be sourced
# during Flume startup.

# Enviroment variables can be set here.

JAVA_HOME=/usr/lib/jvm/java-1.7.0-openjdk-amd64

# Give Flume more memory and pre-allocate, enable remote monitoring via JMX
JAVA_OPTS="-Xms100m -Xmx200m -Dcom.sun.management.jmxremote"

# Note that the Flume conf directory is always included in the classpath.
#FLUME_CLASSPATH=""
\end{lstlisting}

Folgender Befehl startet den Kafka-Server in den Standardeinstellungen:
\begin{verbatim}
bin/kafka-server-start.sh config/server.properties
\end{verbatim}

Die Konfigurationsdatei \textit{flume-conf.properties} kann verändert werden, wenn das Standardprotokollverhalten unerwünscht ist. In der Standardeinstellung wird über einen eigenen Agenten mit Log4j im Verzeichnis \textit{logs} und der Datei \textit{flume.log} protokolliert. Weitere Einstellungen können in der Datei \textit{log4j.properties} vorgenommen werden.

Folgender Aufruf zeigt Optionen für einen Apache Flume-Aufruf aus dem \textit{flume}-Verzeichnis:
\begin{verbatim}
 bin/flume-ng --help
\end{verbatim}

\subsection{Apache Flume Single-Node Beispiel}
\label{section:flumeBeispiel}

Folgendes Beispiel wird an dieser Stelle von der Apache Flume 1.5.0 User Guide \citeint{flume:userGuide} wieder gegeben und wird im Verzeichnis \textit{conf} als \textit{example.conf} abgepeichert.

\begin{lstlisting}[language=BASH, label=lst:flumeExample, caption=Apache Flume Beispiel]
# example.conf: A single-node Flume configuration

# Name the components on this agent
a1.sources = r1
a1.sinks = k1
a1.channels = c1

# Describe/configure the source
a1.sources.r1.type = netcat
a1.sources.r1.bind = localhost
a1.sources.r1.port = 44444

# Describe the sink
a1.sinks.k1.type = logger

# Use a channel which buffers events in memory
a1.channels.c1.type = memory
a1.channels.c1.capacity = 1000
a1.channels.c1.transactionCapacity = 100

# Bind the source and sink to the channel
a1.sources.r1.channels = c1
a1.sinks.k1.channel = c1
\end{lstlisting}

Im Beispiel Listing \ref{lst:flumeExample} wird ein Agent \textit{a1} definiert. In der Quelle \textit{source r1} wird der Linux-Befehl \textit{netcat} auf der internen Netzwerkkarte unter dem Port 44444 als Server-Dienst aktiviert. Damit können beliebige Daten über einen Client wie zum Beispiel \textit{telnet} an den Apache Flume-Agenten gesendet werden. Die Quelle \textit{source r1} leitet die Nachrichten an den Kanal \textit{channel c1} im Zwischenspeicher weiter. In der Sänke \textit{sink c1} wird die Protokollanwendung \textit{log4j} mit dem Bezeichner \textit{logger} aus dem Unterkapitel \ref{section:flumeKonfiguration} gesetzt. Die Ausgabe erfolgt abschließend durch \textit{log4j}-\textit{Appender} (Konsole oder Datei).

Der Apache Flume Agent wird in einer eigenen Shell mit der Beispiel Konfiguration gestartet:
\begin{verbatim}
bin/flume-ng agent --conf conf --conf-file conf/example.conf --name a1
\end{verbatim}

In einer separaten Shell kann eine erfolgreiche Verbindung über \textit{telnet} mit dem TCP-Endpunkt \textit{localhost} Port \textit{44444} aufgebaut, Nachrichten eingegeben und mit dem Bestätigen der Eingabetaste an den Agenten übermittelt werden:
\begin{verbatim}
telnet localhost 44444
\end{verbatim}

Bei einer Eingabe von \textit{Ich bin s40907.} über Telnet auf den Port 44444, wird vom Agenten folgende Nachricht an den LoggerSink ausgegeben:

\begin{lstlisting}[language=BASH, label=lst:flumeLoggerSink, caption=Apache Flume Ausgabe LoggerSink]
21 Jul 2014 20:59:19,820 INFO  [SinkRunner-PollingRunner-DefaultSinkProcessor] (org.apache.flume.sink.LoggerSink.process:70)  - Event: { headers:{} body: 49 63 68 20 62 69 6E 20 73 34 30 39 30 37 0D    Ich bin s40907. }
\end{lstlisting}