\section{Apache Flume}


Nach der Vorstellung von Apache Storm, Kafka und Flume wird in diesem Kapitel Apache S4 vorgestellt. Apache S4 ist eine Abkürzung und steht für Simple Scalable Streaming System und wird von Flavio Junqueira als allgemeine, verteilte, skalierbare, teilweise fehlertolerante, steckbare Plattform bezeichnet \citeint{s4:Proposal}. Zunächst soll eine Kurzübersicht einen ersten Einblick in Apache S4 geben. Anschließend werden die Bewertungskriterien erläutert und vorgestellt.

\begin{table}[tbp]
	\centering
		\begin{tabular}{@{}ll@{}} \toprule
			\textbf{Faktum} & \textbf{Beschreibung} \\ \midrule
			Hauptentwickler & Kishore Gopalakrishna, Flavio Junqueira, Matthieu Morel \\
			& Leo Neumeyer, Bruce Robbins, Daniel Gomez Ferro \\
			Stabile Version & 0.6.0 vom 03.06.2013 \\ 
			Entwicklungsstatus &  Moderat \\
			Entwicklungsversion & 0.7.0 \\
			Sprache & Java \\
			Betriebssystem & plattformunabhängig, benötigt die Java Virtual Machine \\
			& und Apache Zookeeper \\
			Lizenz & Apache License version 2.0 \\
			Webseite & \citeint{s4:home} \\
			Quelltext & \citeint{s4:GitHubApacheMirror} \\			
			\bottomrule			
		\end{tabular}
	\caption{Kurzübersicht Apache S4}
	\label{tab:vors4}
\end{table}